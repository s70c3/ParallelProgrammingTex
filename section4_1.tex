{ %section4_1
	\subsection{Work sequence}
	\begin{enumerate}
		\item Install Linux and the GCC compiler version 4.7.2 or higher on a computer with a multi-core processor. If you cannot install Linux or you do not have a computer with a multi - core processor, you can perform laboratory research on a virtual machine.
		\item Write a console program in C completing task in article 4 (see below). The program cannot use library functions for sorting, performing matrix operations, or calculating statistical values. The program cannot use library functions that are not represented in standard stdio.h, stdlib.h, math.h, sys/time.h header files. You must run the task 50 times with different initial values of the random number generator (RNG). The structure of the program is approximately as follows:
			\begin{figure}[H]
				\lstinputlisting{lab1Example.cpp}
			\end{figure}
		\item Сompile the program without using automatic parallelization following the next command: /home/user/gcc -O3 -Wall -Werror -o lab1-seq lab1.c
		\item Compile the written program using the built-in GCC tool for automatic parallelization (Graphite) with the following command  “/home/user/gcc -O3 -Wall -Werror -floop-parallelize-all -ftree-parallelize-loops=K lab1.c -o lab1-par-K” (assign at least 4 different integer values to the variable K in turn and explain your choice).
		\item The result is one non-parallelized program and four or more parallelized programs.
		\item Close all application programs running on the operating system (including Winamp, uTorrent, browsers, and Skype) so that they do not affect the results of subsequent experiments.
		\itemЗапускать файл lab1-seq из командной строки, увеличивая значения N до значения N1, при котором время выполнения превысит 0.01 с. Подобным образом найти значение N=N2, при котором время выполнения превысит 2 с.
		\itemИспользуя найденные значения N1 и N2, выполнить следующие эксперименты (для автоматизации проведения экспериментов рекомендуется написать скрипт):
			\begin{itemize}
				\itemзапускать lab1-seq для значений \\$N\;=\;{N1,\;N1+\Delta,\;N1+2\Delta,\;N1+3\Delta,…,\;N2}$ и записывать получающиеся значения времени delta\textunderscore ms(N) в функцию $seq(N)$;
				\itemзапускать lab1-par-K для значений \\$N\;=\;{N1,\;N1+\Delta,\;N1+2\Delta,\;N1+3\Delta,…,\;N2}$ и записывать получающиеся значения времени delta\textunderscore ms(N) в функцию $par-K(N)$;
				\itemзначение $\Delta$ выбрать так: $\Delta\;=\;(N2\;-\;N1)/10$.
			\end{itemize}
		\itemНаписать отчёт о проделанной работе.
		\itemПодготовиться к устным вопросам на защите.
		\itemНайти вычислительную сложность алгоритма до и после распараллеливания, сравнить полученные результаты.
		\sloppy
		\item\textbf{Необязательное задание №1 (для получения оценки «четыре» и «пять»).} Провести аналогичные описанным эксперименты, используя вместо gcc компилятор Solaris Studio (или любой другой на своё усмотрение). При компиляции следует использовать следующие опции для автоматического распараллеливания: \verb+«solarisstudio -cc -O3 -xautopar -xloopinfo lab1.c»+.
 		\item\textbf{Необязательное задание №2 (для получения оценки «пять»).} Это задание выполняется только после выполнения предыдущего пункта. Провести аналогичные описанным эксперименты, используя вместо gcc компилятор Intel ICC (или любой другой на своё усмотрение). В ICC следует при компиляции использовать следующие опции для автоматического распараллеливания: \verb+«icc -parallel -par-report -par-threshold K -o lab1-icc-par-K lab1.c»+.
			\parЕсли ключ \verb+«-par-report»+ не работает в вашей версии компилятора, то желательно использовать более актуальный ключ \\\verb+«-qopt-report-phase=par»+.
	\end{enumerate}
	
}